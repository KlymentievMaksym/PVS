\documentclass{article}

\usepackage[utf8]{inputenc}
\usepackage[english, ukrainian]{babel}
\usepackage[fontsize = 14pt]{fontsize}
\usepackage{fontspec}
\setmainfont{Times New Roman}
\setmonofont{Courier New}

\usepackage[svgnames]{xcolor}

\usepackage{geometry}
\usepackage{float}
\geometry{left=25mm,right=15mm,top=20mm,bottom=20mm}

\usepackage{geometry}
\usepackage{amsthm}
\usepackage{amsfonts}
\usepackage{graphicx}
\usepackage[ruled]{algorithm2e}
\usepackage{hyperref}
\usepackage{biblatex}
\usepackage{csquotes}
\usepackage{mathtools}
\usepackage{amsmath}
\usepackage{amssymb}
\usepackage{bbm}
\usepackage{tabularx}
% \usepackage{xcolor}

\usepackage{enumitem}
\usepackage{nicefrac}

\usepackage{listings}
\definecolor{codegreen}{rgb}{0,0.6,0}
\definecolor{codegray}{rgb}{0.5,0.5,0.5}
\definecolor{codepurple}{rgb}{0.58,0,0.82}
\definecolor{backcolour}{rgb}{0.95,0.95,0.92}

\lstdefinestyle{mystyle}{
    backgroundcolor=\color{backcolour},   
    commentstyle=\color{codegreen},
    keywordstyle=\color{magenta},
    % numberstyle=\tiny\color{codegray},
    stringstyle=\color{codepurple},
    basicstyle=\ttfamily\footnotesize,
    breakatwhitespace=false,         
    breaklines=true,                 
    captionpos=b,                    
    keepspaces=true,                 
    % numbers=left,                    
    % numbersep=5pt,                  
    showspaces=false,                
    showstringspaces=false,
    showtabs=false,                  
    tabsize=2
}

\hypersetup{colorlinks=true, linkcolor=[RGB]{255, 3, 209}, citecolor={black}}
\lstset{style=mystyle}

\graphicspath{ {../Images/} }

\begin{document}
    \begin{titlepage}
        \begin{center}

        Міністерство освіти і науки України
        
        НТУУ «Київський політехнічний інститут»
        
        Фізико-технічний інститут
        \vspace{3.3cm}
        
        {\textbf{Проектування високонавантажених систем}\\Лабораторна робота No6\\Кешування з використанням Hazelcast}

        \vspace{10cm}

        \begin{flushright}
            \textbf{Виконав:}\\Студент 4-го курсу\\групи ФІ-21\\Климентьєв Максим\\
            \textbf{Перевірив:}\\\text{\_\_\_\_\_\_\_\_\_\_\_\_\_\_\_\_\_\_}
        \end{flushright}

        \end{center}
    \end{titlepage}
    \newpage

    \pagenumbering{gobble}
    \tableofcontents
    \cleardoublepage
    \pagenumbering{arabic}
    \setcounter{page}{3}

    \newpage
    \section{Код реалізації}
    \lstinputlisting[title=Counter.py, language=python]{../Counter.py}
    \lstinputlisting[title=CounterMapStore.java, language=java]{../CounterMapStore.java}
    \lstinputlisting[title=docker-compose.yaml, style=yaml]{../docker-compose.yaml}
    \lstinputlisting[title=Dockerfile, language=Dockerfile]{../Dockerfile}
    \lstinputlisting[title=hazelcast.yaml, style=yaml]{../hazelcast.yaml}

    \newpage
    \section{Результати}
        На основі першого завдання з Web-counter, додати до нього Read-through та Write-through кешування.

        У якості системи кешування, використати кластер Hazelcast з відповідними налаштуваннями https://docs.hazelcast.com/hazelcast/5.6/mapstore/working-with-external-data

        Ідея полягає у тому, що значення каунтера зберігається у базі даних, але операції з ним відбувались не на пряму, а крізь кеш.

        % \begin{lstlisting}[language=python]

        % \end{lstlisting}

        \begin{enumerate}
            \item Налаштувати Read-through та Write-through на основі generic MapLoader та generic MapStore для БД PostgreSQL (або MySQL)
                % \begin{lstlisting}[language=python]

                % \end{lstlisting}
            \item Припускаючи, що у нас буде 4 різних каунтерів, додати в Hazelcast Map, яка відповідає таблиці БД, 4 окремих записи з різними ID (1-4) та 0 у якості початкового значення. Перевірити через прямий запит до БД, що ці записи в ній зʼявились 
                % \begin{lstlisting}[language=python]

                % \end{lstlisting}
            \item Для окремих створених ключів (каунтерів) повторити тести 1) - 4) з першої роботи, забезпечуючи коректність результатів (тобто щоб кінцевий результат був коректним) та міряючи час. Тобто тест 1 має використовувати каунтер 1, тест 2 - каунтер 2, і т.д. 
                \begin{enumerate}[label=(\arabic*)]
                    \item Один клієнт робить послідовно 10К викликів, кінцеве значення count = 10K - порахувати кількість запитів в секунду
                    \item Два клієнти одночасно роблять по 10К викликів кожен, кінцеве значення count = 20K - порахувати кількість запитів в секунду
                    \item 5 клієнтів одночасно роблять по 10К викликів кожен, кінцеве значення count = 50K - порахувати кількість запитів в секунду
                    \item 10 клієнтів одночасно роблять по 10К викликів кожен, кінцеве значення count = 100K - порахувати кількість запитів в секунду
                \end{enumerate}
                % \begin{lstlisting}[language=python]

                % \end{lstlisting}
            \item Перевірити кінцеві значення каунтерів у БД та порівняти отримані результати з продуктивності з отриманими у першій роботі. 
                \begin{lstlisting}[language=sql, style=log]
[INFO] Connecting to Hazelcast...
[INFO] Clearing Counters...
[INFO] Data: [(1, 0), (2, 0), (3, 0), (4, 0)]
[INFO] Running Counter 1
[INFO] Counter ID: 1
[INFO] Clients: 1 | Increments per client: 10000
[RESULT] Time: 44.36 sec
[RESULT] Throughput: 225.41
[RESULT] Received: 10_000 (100.00% from Expected: 10_000)
[INFO] Running Counter 2
[INFO] Counter ID: 2
[INFO] Clients: 2 | Increments per client: 10000
[RESULT] Time: 69.05 sec
[RESULT] Throughput: 289.64
[RESULT] Received: 20_000 (100.00% from Expected: 20_000)
[INFO] Running Counter 3
[INFO] Counter ID: 3
[INFO] Clients: 5 | Increments per client: 10000
[RESULT] Time: 170.70 sec
[RESULT] Throughput: 292.91
[RESULT] Received: 50_000 (100.00% from Expected: 50_000)
[INFO] Running Counter 4
[INFO] Counter ID: 4
[INFO] Clients: 10 | Increments per client: 10000
[RESULT] Time: 354.13 sec
[RESULT] Throughput: 282.39
[RESULT] Received: 100_000 (100.00% from Expected: 100_000)
[INFO] Data: [(1, 10000), (2, 20000), (3, 50000), (4, 100000)]
                \end{lstlisting}
                \begin{lstlisting}[language=sql, style=log]
[INFO] Один клієнт
[Name] Memory
[Sent] 10000 requests in 3.29 seconds
[Throughput]: 3040.52 requests/sec
[Name] DataBase
[Sent] 10000 requests in 10.75 seconds
[Throughput]: 929.95 requests/sec
[Name] Hazelcast Cash
[RESULT] Time: 44.36 sec
[RESULT] Throughput: 225.41

[INFO] Два клієнти
[Name] Memory
[Sent] 20000 requests in 6.67 seconds
[Throughput]: 2996.28 requests/sec
[Name] DataBase
[Sent] 20000 requests in 22.13 seconds
[Throughput]: 903.66 requests/sec
[Name] Hazelcast Cash
[RESULT] Time: 69.05 sec
[RESULT] Throughput: 289.64

[INFO] П''ять клієнтів
[Name] Memory
[Sent] 50000 requests in 17.83 seconds
[Throughput]: 2804.70 requests/sec
[Name] DataBase
[Sent] 50000 requests in 57.78 seconds
[Throughput]: 865.40 requests/sec
[Name] Hazelcast Cash
[RESULT] Time: 170.70 sec
[RESULT] Throughput: 292.91

[INFO] Десять клієнтів
\textbf{}
[Name] Memory
[Sent] 100000 requests in 39.65 seconds
[Throughput]: 2521.88 requests/sec
[Name] DataBase
[Sent] 100000 requests in 117.59 seconds
[Throughput]: 850.39 requests/sec
[Name] Hazelcast Cash
[RESULT] Time: 354.13 sec
[RESULT] Throughput: 282.39
                \end{lstlisting}
                \textbf{Виходить повільний запис, але надійний та можна дуже швидко читати}
        \end{enumerate}


\end{document}
