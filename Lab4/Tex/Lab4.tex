\documentclass{article}

\usepackage[utf8]{inputenc}
\usepackage[english, ukrainian]{babel}
\usepackage[fontsize = 14pt]{fontsize}
\usepackage{fontspec}
\setmainfont{Times New Roman}
\setmonofont{Courier New}

\usepackage[svgnames]{xcolor}

\usepackage{geometry}
\usepackage{float}
\geometry{left=25mm,right=15mm,top=20mm,bottom=20mm}

\usepackage{geometry}
\usepackage{amsthm}
\usepackage{amsfonts}
\usepackage{graphicx}
\usepackage[ruled]{algorithm2e}
\usepackage{hyperref}
\usepackage{biblatex}
\usepackage{csquotes}
\usepackage{mathtools}
\usepackage{amsmath}
\usepackage{amssymb}
\usepackage{bbm}
\usepackage{tabularx}
% \usepackage{xcolor}

\usepackage{enumitem}
\usepackage{nicefrac}

\usepackage{listings}
\definecolor{codegreen}{rgb}{0,0.6,0}
\definecolor{codegray}{rgb}{0.5,0.5,0.5}
\definecolor{codepurple}{rgb}{0.58,0,0.82}
\definecolor{backcolour}{rgb}{0.95,0.95,0.92}

\lstdefinestyle{mystyle}{
    backgroundcolor=\color{backcolour},   
    commentstyle=\color{codegreen},
    keywordstyle=\color{magenta},
    % numberstyle=\tiny\color{codegray},
    stringstyle=\color{codepurple},
    basicstyle=\ttfamily\footnotesize,
    breakatwhitespace=false,         
    breaklines=true,                 
    captionpos=b,                    
    keepspaces=true,                 
    % numbers=left,                    
    % numbersep=5pt,                  
    showspaces=false,                
    showstringspaces=false,
    showtabs=false,                  
    tabsize=2
}

\hypersetup{colorlinks=true, linkcolor=[RGB]{255, 3, 209}, citecolor={black}}
\lstset{style=mystyle}

\graphicspath{ {../Images/} }

\begin{document}
    \begin{titlepage}
        \begin{center}

        Міністерство освіти і науки України
        
        НТУУ «Київський політехнічний інститут»
        
        Фізико-технічний інститут
        \vspace{3.3cm}
        
        {\textbf{Проектування високонавантажених систем}\\Лабораторна робота No4\\Налаштування реплікації та перевірка відмовостійкості MongoDB}

        \vspace{10cm}

        \begin{flushright}
            \textbf{Виконав:}\\Студент 4-го курсу\\групи ФІ-21\\Климентьєв Максим\\
            \textbf{Перевірив:}\\\text{\_\_\_\_\_\_\_\_\_\_\_\_\_\_\_\_\_\_}
        \end{flushright}

        \end{center}
    \end{titlepage}
    \newpage

    \pagenumbering{gobble}
    \tableofcontents
    \cleardoublepage
    \pagenumbering{arabic}
    \setcounter{page}{3}

    \newpage
    \section{Код реалізації}

    \newpage
    \section{Результати}

    I Налаштування реплікації
    \begin{enumerate}
        \item Налаштувати реплікацію в конфігурації: Primary with Two Secondary Members (P-S-S) (всі ноди можуть бути запущені як окремі процеси або у Docker контейнерах)
            \begin{lstlisting}[language=python]
replicaset [direct: primary] test> rs.status()
{
    set: 'replicaset',
    date: ISODate('2025-12-14T14:45:33.199Z'),
    myState: 1,
    term: Long('1'),
    syncSourceHost: '',
    syncSourceId: -1,
    heartbeatIntervalMillis: Long('2000'),
    majorityVoteCount: 2,
    writeMajorityCount: 2,
    votingMembersCount: 3,
    writableVotingMembersCount: 3,
    optimes: {
        lastCommittedOpTime: { ts: Timestamp({ t: 1765723531, i: 1 }), t: Long('1') },
        lastCommittedWallTime: ISODate('2025-12-14T14:45:31.010Z'),
        readConcernMajorityOpTime: { ts: Timestamp({ t: 1765723531, i: 1 }), t: Long('1') },
        appliedOpTime: { ts: Timestamp({ t: 1765723531, i: 1 }), t: Long('1') },
        durableOpTime: { ts: Timestamp({ t: 1765723531, i: 1 }), t: Long('1') },
        writtenOpTime: { ts: Timestamp({ t: 1765723531, i: 1 }), t: Long('1') },
        lastAppliedWallTime: ISODate('2025-12-14T14:45:31.010Z'),
        lastDurableWallTime: ISODate('2025-12-14T14:45:31.010Z'),
        lastWrittenWallTime: ISODate('2025-12-14T14:45:31.010Z')
    },
    lastStableRecoveryTimestamp: Timestamp({ t: 1765723511, i: 1 }),
    electionCandidateMetrics: {
        lastElectionReason: 'electionTimeout',
        lastElectionDate: ISODate('2025-12-14T14:35:30.954Z'),
        electionTerm: Long('1'),
        lastCommittedOpTimeAtElection: { ts: Timestamp({ t: 1765722919, i: 1 }), t: Long('-1') },
        lastSeenWrittenOpTimeAtElection: { ts: Timestamp({ t: 1765722919, i: 1 }), t: Long('-1') },
        lastSeenOpTimeAtElection: { ts: Timestamp({ t: 1765722919, i: 1 }), t: Long('-1') },
        numVotesNeeded: 2,
        priorityAtElection: 2,
        electionTimeoutMillis: Long('10000'),
        numCatchUpOps: Long('0'),
        newTermStartDate: ISODate('2025-12-14T14:35:31.003Z'),
        wMajorityWriteAvailabilityDate: ISODate('2025-12-14T14:35:31.473Z')
    },
    members: [
        {
        _id: 0,
        name: 'mongo1:27017',
        health: 1,
        state: 1,
        stateStr: 'PRIMARY',
        uptime: 620,
        optime: { ts: Timestamp({ t: 1765723531, i: 1 }), t: Long('1') },
        optimeDate: ISODate('2025-12-14T14:45:31.000Z'),
        optimeWritten: { ts: Timestamp({ t: 1765723531, i: 1 }), t: Long('1') },
        optimeWrittenDate: ISODate('2025-12-14T14:45:31.000Z'),
        lastAppliedWallTime: ISODate('2025-12-14T14:45:31.010Z'),
        lastDurableWallTime: ISODate('2025-12-14T14:45:31.010Z'),
        lastWrittenWallTime: ISODate('2025-12-14T14:45:31.010Z'),
        syncSourceHost: '',
        syncSourceId: -1,
        infoMessage: '',
        electionTime: Timestamp({ t: 1765722930, i: 1 }),
        electionDate: ISODate('2025-12-14T14:35:30.000Z'),
        configVersion: 1,
        configTerm: 1,
        self: true,
        lastHeartbeatMessage: ''
        },
        {
        _id: 1,
        name: 'mongo2:27017',
        health: 1,
        state: 2,
        stateStr: 'SECONDARY',
        uptime: 613,
        optime: { ts: Timestamp({ t: 1765723531, i: 1 }), t: Long('1') },
        optimeDurable: { ts: Timestamp({ t: 1765723531, i: 1 }), t: Long('1') },
        optimeWritten: { ts: Timestamp({ t: 1765723531, i: 1 }), t: Long('1') },
        optimeDate: ISODate('2025-12-14T14:45:31.000Z'),
        optimeDurableDate: ISODate('2025-12-14T14:45:31.000Z'),
        optimeWrittenDate: ISODate('2025-12-14T14:45:31.000Z'),
        lastAppliedWallTime: ISODate('2025-12-14T14:45:31.010Z'),
        lastDurableWallTime: ISODate('2025-12-14T14:45:31.010Z'),
        lastWrittenWallTime: ISODate('2025-12-14T14:45:31.010Z'),
        lastHeartbeat: ISODate('2025-12-14T14:45:32.968Z'),
        lastHeartbeatRecv: ISODate('2025-12-14T14:45:31.970Z'),
        pingMs: Long('0'),
        lastHeartbeatMessage: '',
        syncSourceHost: 'mongo1:27017',
        syncSourceId: 0,
        infoMessage: '',
        configVersion: 1,
        configTerm: 1
        },
        {
        _id: 2,
        name: 'mongo3:27017',
        health: 1,
        state: 2,
        stateStr: 'SECONDARY',
        uptime: 613,
        optime: { ts: Timestamp({ t: 1765723531, i: 1 }), t: Long('1') },
        optimeDurable: { ts: Timestamp({ t: 1765723531, i: 1 }), t: Long('1') },
        optimeWritten: { ts: Timestamp({ t: 1765723531, i: 1 }), t: Long('1') },
        optimeDate: ISODate('2025-12-14T14:45:31.000Z'),
        optimeDurableDate: ISODate('2025-12-14T14:45:31.000Z'),
        optimeWrittenDate: ISODate('2025-12-14T14:45:31.000Z'),
        lastAppliedWallTime: ISODate('2025-12-14T14:45:31.010Z'),
        lastDurableWallTime: ISODate('2025-12-14T14:45:31.010Z'),
        lastWrittenWallTime: ISODate('2025-12-14T14:45:31.010Z'),
        lastHeartbeat: ISODate('2025-12-14T14:45:32.969Z'),
        lastHeartbeatRecv: ISODate('2025-12-14T14:45:31.970Z'),
        pingMs: Long('0'),
        lastHeartbeatMessage: '',
        syncSourceHost: 'mongo1:27017',
        syncSourceId: 0,
        infoMessage: '',
        configVersion: 1,
        configTerm: 1
        }
    ],
    ok: 1,
    '$clusterTime': {
        clusterTime: Timestamp({ t: 1765723531, i: 1 }),
        signature: {
        hash: Binary.createFromBase64('AAAAAAAAAAAAAAAAAAAAAAAAAAA=', 0),
        keyId: Long('0')
        }
    },
    operationTime: Timestamp({ t: 1765723531, i: 1 })
    }
            \end{lstlisting}
        \item Спробувати зробити запис з однією відключеною нодою та write concern рівнім 3 та нескінченім таймаутом. Спробувати під час таймаута включити відключену ноду
            \begin{lstlisting}
[INFO] [PRIMARY] Starting 1 client with w=3 write concern and no timeout...
[INFO] [START RESULT] Right before stoping PRIMARY...
{'_id': '1', 'likes': 0}
docker stop mongo1
mongo1
[INFO] [MID RESULT] Right after stoping PRIMARY...
[TIMEOUT] ServerSelectionTimeoutError
[TIMEOUT] ServerSelectionTimeoutError
[INFO] [MID RESULT] After 10 seconds stoping PRIMARY...
[TIMEOUT] ServerSelectionTimeoutError
docker start mongo1
mongo1
[INFO] [FINAL RESULT]
{'_id': '1', 'likes': 0}

[INFO] [SECONDARY] Starting 1 client with w=3 write concern and no timeout...
[INFO] [START RESULT] Right before stoping SECONDARY...
{'_id': '1', 'likes': 0}
docker stop mongo3
mongo3
[INFO] [MID RESULT] Right after stoping SECONDARY...
{'_id': '1', 'likes': 0}
[INFO] [MID RESULT] After 10 seconds stoping SECONDARY...
{'_id': '1', 'likes': 1}
docker start mongo3
mongo3
[INFO] [FINAL RESULT]
{'_id': '1', 'likes': 1}

            \end{lstlisting}
        \item Аналогічно попередньому пункту, але задати скінченний таймаут та дочекатись його закінчення. Перевірити чи данні записались і чи доступні на читання з рівнем readConcern: “majority”
            \begin{lstlisting}
[INFO] [PRIMARY] Starting 1 client with w=3 write concern and 'majority' read concern, but timeout = 3000...
[INFO] [START RESULT] Right before stoping PRIMARY...
{'_id': '1', 'likes': 1}
docker stop mongo1
mongo1
[INFO] [MID RESULT] Right after stoping PRIMARY...
[TIMEOUT] ServerSelectionTimeoutError
[TIMEOUT] ServerSelectionTimeoutError
[INFO] [MID RESULT] After 10 seconds stoping PRIMARY...
[TIMEOUT] ServerSelectionTimeoutError
docker start mongo1
mongo1
[INFO] [FINAL RESULT]
{'_id': '1', 'likes': 1}

[INFO] [SECONDARY] Starting 1 client with w=3 write concern and 'majority' read concern, but timeout = 3000...
[INFO] [START RESULT] Right before stoping SECONDARY...
{'_id': '1', 'likes': 1}
docker stop mongo3
mongo3
[INFO] [MID RESULT] Right after stoping SECONDARY...
{'_id': '1', 'likes': 1}
[TIMEOUT] WTimeoutError
[INFO] [MID RESULT] After 10 seconds stoping SECONDARY...
{'_id': '1', 'likes': 2}
docker start mongo3
mongo3
[INFO] [FINAL RESULT]
{'_id': '1', 'likes': 2}
            \end{lstlisting}
        \item Продемонстрував перевибори primary node відключивши поточний primary
            \begin{lstlisting}

            \end{lstlisting}
        \item і що після відновлення роботи старої primary на неї реплікуються нові дані, які з'явилися під час її простою
            \begin{lstlisting}

            \end{lstlisting}
    \end{enumerate}

    II Аналіз продуктивності та перевірка цілісності

    Cтворити колекцію \(таблицю\) з каунтером лайків. Далі з 10 окремих клієнтів одночасно запустити інкерементацію каунтеру лайків по 10000 на кожного клієнта з різними опціями взаємодії з MongoDB.

    \begin{enumerate}
        \item Вказавши у парметрах findOneAndUpdate writeConcern = 1
            \begin{lstlisting}

            \end{lstlisting}
        \item Вказавши у парметрах findOneAndUpdate writeConcern = majority
            \begin{lstlisting}

            \end{lstlisting}
        \item Повторно запустить код при writeConcern = 1, але тепер під час роботи відключіть Primary ноду і подивитись що буде обрана інша Primary нода, яка продовжить обробку запитів, і чи кінцевий результат буде коректним
            \begin{lstlisting}

            \end{lstlisting}
        \item Повторно запустить код при writeConcern = majority, але тепер під час роботи відключіть Primary ноду і подивитись що буде обрана інша Primary нода, яка продовжить обробку запитів, і чи кінцевий результат буде коректним
            \begin{lstlisting}

            \end{lstlisting}
    \end{enumerate}

\end{document}