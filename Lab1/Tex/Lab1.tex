\documentclass{article}

\usepackage[utf8]{inputenc}
\usepackage[english, ukrainian]{babel}
\usepackage[fontsize = 14pt]{fontsize}
\usepackage{fontspec}
\setmainfont{Times New Roman}
\setmonofont{Courier New}

\usepackage[svgnames]{xcolor}

\usepackage{geometry}
\usepackage{float}
\geometry{left=25mm,right=15mm,top=20mm,bottom=20mm}

\usepackage{geometry}
\usepackage{amsthm}
\usepackage{amsfonts}
\usepackage{graphicx}
\usepackage[ruled]{algorithm2e}
\usepackage{hyperref}
\usepackage{biblatex}
\usepackage{csquotes}
\usepackage{mathtools}
\usepackage{amsmath}
\usepackage{amssymb}
\usepackage{bbm}
\usepackage{tabularx}
% \usepackage{xcolor}

\usepackage{enumitem}
\usepackage{nicefrac}

\usepackage{listings}
\definecolor{codegreen}{rgb}{0,0.6,0}
\definecolor{codegray}{rgb}{0.5,0.5,0.5}
\definecolor{codepurple}{rgb}{0.58,0,0.82}
\definecolor{backcolour}{rgb}{0.95,0.95,0.92}

\lstdefinestyle{mystyle}{
    backgroundcolor=\color{backcolour},   
    commentstyle=\color{codegreen},
    keywordstyle=\color{magenta},
    % numberstyle=\tiny\color{codegray},
    stringstyle=\color{codepurple},
    basicstyle=\ttfamily\footnotesize,
    breakatwhitespace=false,         
    breaklines=true,                 
    captionpos=b,                    
    keepspaces=true,                 
    % numbers=left,                    
    % numbersep=5pt,                  
    showspaces=false,                
    showstringspaces=false,
    showtabs=false,                  
    tabsize=2
}

\hypersetup{colorlinks=true, linkcolor=[RGB]{255, 3, 209}, citecolor={black}}
\lstset{style=mystyle}

\graphicspath{ {../Images/} }

\begin{document}
    \begin{titlepage}
        \begin{center}

        Міністерство освіти і науки України
        
        НТУУ «Київський політехнічний інститут»
        
        Фізико-технічний інститут
        \vspace{3.3cm}
        
        {\textbf{Проектування високонавантажених систем}\\Лабораторна робота No1\\Web-counter}

        \vspace{10cm}

        \begin{flushright}
            \textbf{Виконав:}\\Студент 4-го курсу\\групи ФІ-21\\Климентьєв Максим\\
            \textbf{Перевірив:}\\\text{\_\_\_\_\_\_\_\_\_\_\_\_\_\_\_\_\_\_}
        \end{flushright}

        \end{center}
    \end{titlepage}
    \newpage

    \pagenumbering{gobble}
    \tableofcontents
    \cleardoublepage
    \pagenumbering{arabic}
    \setcounter{page}{3}

    \newpage
    \section{Код реалізації}
    \begin{lstlisting}[language=Python, title=Counter.py]
from threading import Lock
import sqlite3 as sql
import os

path_to_root = os.getcwd()
if "Lab1" not in path_to_root:
    path_to_root += "\\Lab1"
if "Data" not in path_to_root:
    path_to_root += "\\Data\\"


class Counter:
    def __init__(self):
        self._lock = Lock()

    def inc_count(self):
        raise NotImplementedError

    def get_count(self):
        raise NotImplementedError


class CounterMem(Counter):
    def __init__(self):
        super().__init__()
        self.value = 0

    def inc_count(self):
        with self._lock:
            self.value += 1
            return self.value

    def get_count(self):
        with self._lock:
            return self.value


class CounterDB(Counter):
    def __init__(self, db_path: str = path_to_root + "\\data.db"):
        super().__init__()
        self.db_path = db_path
        self.conn = sql.connect(db_path, check_same_thread=False)
        cur = self.conn.cursor()
        cur.execute("""
            CREATE TABLE IF NOT EXISTS counters (
                id INTEGER PRIMARY KEY,
                value INTEGER NOT NULL
            )
        """)
        cur.execute("INSERT OR IGNORE INTO counters (id, value) VALUES (1, 0)")
        cur.execute("UPDATE counters SET value = 0")
        self.conn.commit()

    def inc_count(self):
        with self._lock:
            cur = self.conn.cursor()
            cur.execute("UPDATE counters SET value = value + 1")
            self.conn.commit()
            cur.execute("SELECT value FROM counters")
            return cur.fetchone()[0]

    def get_count(self):
        with self._lock:
            cur = self.conn.cursor()
            cur.execute("SELECT value FROM counters")
            return cur.fetchone()[0]
    \end{lstlisting}
    \begin{lstlisting}[language=Python, title=WebCounter.py]
from fastapi import FastAPI
import uvicorn

from Counter import CounterMem, CounterDB


class WebCounterApp:
    def __init__(self, counter: str = 'mem'):
        wrong_name = NameError("Wrong Name. Expected one from ['mem', 'memory', 'db', 'database']")
        self.app = FastAPI(title="Web Counter")
        self.counter = CounterMem() if counter.lower() == 'mem' or counter.lower() == 'memory' else CounterDB() if counter.lower() == 'db' or counter.lower() == 'database' else wrong_name
        if self.counter is wrong_name:
            raise self.counter
        self._setup_routes()

    def _setup_routes(self):
        @self.app.get("/inc")
        def inc_count():
            value = self.counter.inc_count()
            return value

        @self.app.get("/cnt")
        def get_count():
            value = self.counter.get_count()
            return value

    def run(self, host: str = "localhost", port: int = 8000, log_level: str = None):
        uvicorn.run(self.app, host=host, port=port, log_level=log_level)


if __name__ == "__main__":
    web_app = WebCounterApp('db')
    web_app.run()

    \end{lstlisting}
    \begin{lstlisting}[language=Python, title=Client.py]
import asyncio
import aiohttp
import time


class HttpClient:
    def __init__(self, base_url: str, concurrency: int = 100):
        self.base_url = base_url.rstrip('/')
        self.concurrency = concurrency

    async def _fetch(self, session: aiohttp.ClientSession, endpoint: str):
        async with session.get(f"{self.base_url}/{endpoint}") as resp:
            return await resp.text()

    async def _worker(self, session, endpoint, num_requests):
        tasks = [self._fetch(session, endpoint) for _ in range(num_requests)]
        await asyncio.gather(*tasks)

    async def run_load_test(self, endpoint: str, total_requests: int, name: str = "Memory"):
        start_time = time.perf_counter()
        async with aiohttp.ClientSession() as session:
            per_worker = total_requests // self.concurrency
            tasks = [
                self._worker(session, endpoint, per_worker)
                for _ in range(self.concurrency)
            ]
            await asyncio.gather(*tasks)

            final_value = await self._fetch(session, "cnt")
        duration = time.perf_counter() - start_time
        throughput = total_requests / duration
        print(f"[Name] {name}")
        print(f"[Clients|Requests] {per_worker} requests each for {self.concurrency} clients")
        print(f"[Sent] {total_requests} requests in {duration:.2f} seconds")
        print(f"[Count] {final_value}")
        print(f"[Throughput]: {throughput:.2f} requests/sec")
        print()
        return throughput


if __name__ == "__main__":
    client = HttpClient("http://localhost:8000", concurrency=200)
    asyncio.run(client.run_load_test("inc", total_requests=10_000))

    \end{lstlisting}
    \begin{lstlisting}[language=Python, title=main.py]
import time

import multiprocessing
import asyncio

from WebCounter import WebCounterApp
from Client import HttpClient

def run_server(counter_type, port):
    app = WebCounterApp(counter_type)
    app.run(port=port, log_level="critical")

if __name__ == "__main__":
    clients_amounts = [1, 2, 5, 10]
    requests_amount = 10_000


    for clients_amount in clients_amounts:
        total_requests = requests_amount * clients_amount
        # -------------------------------------------------- #

        p1 = multiprocessing.Process(target=run_server, args=('mem', 8000))
        p1.start()

        time.sleep(1)

        client = HttpClient("http://localhost:8000", concurrency=clients_amount)
        asyncio.run(client.run_load_test("inc", total_requests=total_requests, name="Memory"))

        p1.terminate()
        p1.join()

        # -------------------------------------------------- #

        p2 = multiprocessing.Process(target=run_server, args=('db', 8000))
        p2.start()

        time.sleep(1)


        client = HttpClient("http://localhost:8000", concurrency=clients_amount)
        asyncio.run(client.run_load_test("inc", total_requests=total_requests, name="DataBase"))

        p2.terminate()
        p2.join()

    \end{lstlisting}

    \newpage
    \section{Результати}

    \textbf{Один клієнт}\newline
    [Name] Memory\newline
    [Clients|Requests] 10000 requests each for 1 clients\newline
    [Sent] 10000 requests in 3.29 seconds\newline
    [Count] 10000\newline
    [Throughput]: 3040.52 requests/sec\newline
\newline
    [Name] DataBase\newline
    [Clients|Requests] 10000 requests each for 1 clients\newline
    [Sent] 10000 requests in 10.75 seconds\newline
    [Count] 10000\newline
    [Throughput]: 929.95 requests/sec\newline
\newline
    \textbf{Два клієнти}\newline
    [Name] Memory\newline
    [Clients|Requests] 10000 requests each for 2 clients\newline
    [Sent] 20000 requests in 6.67 seconds\newline
    [Count] 20000\newline
    [Throughput]: 2996.28 requests/sec\newline
\newline
    [Name] DataBase\newline
    [Clients|Requests] 10000 requests each for 2 clients\newline
    [Sent] 20000 requests in 22.13 seconds\newline
    [Count] 20000\newline
    [Throughput]: 903.66 requests/sec\newline
\newline
    \textbf{П'ять клієнтів}\newline
    [Name] Memory\newline
    [Clients|Requests] 10000 requests each for 5 clients\newline
    [Sent] 50000 requests in 17.83 seconds\newline
    [Count] 50000\newline
    [Throughput]: 2804.70 requests/sec\newline
\newline
    [Name] DataBase\newline
    [Clients|Requests] 10000 requests each for 5 clients\newline
    [Sent] 50000 requests in 57.78 seconds\newline
    [Count] 50000\newline
    [Throughput]: 865.40 requests/sec\newline
\newline
    \textbf{Десять клієнтів}\newline
    [Name] Memory\newline
    [Clients|Requests] 10000 requests each for 10 clients\newline
    [Sent] 100000 requests in 39.65 seconds\newline
    [Count] 100000\newline
    [Throughput]: 2521.88 requests/sec\newline
\newline
    [Name] DataBase\newline
    [Clients|Requests] 10000 requests each for 10 clients\newline
    [Sent] 100000 requests in 117.59 seconds\newline
    [Count] 100000\newline
    [Throughput]: 850.39 requests/sec\newline
\end{document}