\documentclass{article}

\usepackage[utf8]{inputenc}
\usepackage[english, ukrainian]{babel}
\usepackage[fontsize = 14pt]{fontsize}
\usepackage{fontspec}
\setmainfont{Times New Roman}
\setmonofont{Courier New}

\usepackage[svgnames]{xcolor}

\usepackage{geometry}
\usepackage{float}
\geometry{left=25mm,right=15mm,top=20mm,bottom=20mm}

\usepackage{geometry}
\usepackage{amsthm}
\usepackage{amsfonts}
\usepackage{graphicx}
\usepackage[ruled]{algorithm2e}
\usepackage{hyperref}
\usepackage{biblatex}
\usepackage{csquotes}
\usepackage{mathtools}
\usepackage{amsmath}
\usepackage{amssymb}
\usepackage{bbm}
\usepackage{tabularx}
% \usepackage{xcolor}

\usepackage{enumitem}
\usepackage{nicefrac}

\usepackage{listings}
\definecolor{codegreen}{rgb}{0,0.6,0}
\definecolor{codegray}{rgb}{0.5,0.5,0.5}
\definecolor{codepurple}{rgb}{0.58,0,0.82}
\definecolor{backcolour}{rgb}{0.95,0.95,0.92}

\lstdefinestyle{mystyle}{
    backgroundcolor=\color{backcolour},   
    commentstyle=\color{codegreen},
    keywordstyle=\color{magenta},
    % numberstyle=\tiny\color{codegray},
    stringstyle=\color{codepurple},
    basicstyle=\ttfamily\footnotesize,
    breakatwhitespace=false,         
    breaklines=true,                 
    captionpos=b,                    
    keepspaces=true,                 
    % numbers=left,                    
    % numbersep=5pt,                  
    showspaces=false,                
    showstringspaces=false,
    showtabs=false,                  
    tabsize=2
}

\hypersetup{colorlinks=true, linkcolor=[RGB]{255, 3, 209}, citecolor={black}}
\lstset{style=mystyle}

\graphicspath{ {../Images/} }

\begin{document}
    \begin{titlepage}
        \begin{center}

        Міністерство освіти і науки України
        
        НТУУ «Київський політехнічний інститут»
        
        Фізико-технічний інститут
        \vspace{3.3cm}
        
        {\textbf{Проектування високонавантажених систем}\\Лабораторна робота No7\\Налаштування шардінгу в MongoDB}

        \vspace{10cm}

        \begin{flushright}
            \textbf{Виконав:}\\Студент 4-го курсу\\групи ФІ-21\\Климентьєв Максим\\
            \textbf{Перевірив:}\\\text{\_\_\_\_\_\_\_\_\_\_\_\_\_\_\_\_\_\_}
        \end{flushright}

        \end{center}
    \end{titlepage}
    \newpage

    \pagenumbering{gobble}
    \tableofcontents
    \cleardoublepage
    \pagenumbering{arabic}
    \setcounter{page}{3}

    \newpage
    \section{Код реалізації}
    \newpage
    \section{Результати}

        \begin{enumerate}
            \item Сконфігуруйте 3 інстанси (сервери) MongoDB у якості шард. Ці шарди потім будуть звʼязані з Zones (https://www.mongodb.com/docs/manual/core/zone-sharding/)
                \begin{lstlisting}[language=python]

                \end{lstlisting}
                \begin{itemize}
                    \item Ranged Sharding (with Zone Ranges)
                        \begin{lstlisting}[language=python]

                        \end{lstlisting}
                    \item Zones
                        \begin{lstlisting}[language=python]

                        \end{lstlisting}
                \end{itemize}
            \item Створіть колекцію Users
                \begin{lstlisting}[language=python]

                \end{lstlisting}
        \end{enumerate}

        Окрім інших атрибутів, записи User мають містити атрибут region, який може приймати одне з трьох значень region = {EU, USA, Asia}
            \begin{lstlisting}[language=python]

            \end{lstlisting}

        \begin{enumerate}
            \item Сконфігуруйте шардінг таким чином, щоб кожна з шард була проасоційована з одним з регіоном: Shard 1 - EU, Shard 2 - USA, Shard 3 - Asia
                \begin{lstlisting}[language=python]

                \end{lstlisting}
        \end{enumerate}
        І записи які будуть вставлятись в колекцію Users мають зберігатись на відповідній шарді в залежності від значення region
            \begin{lstlisting}[language=python]

            \end{lstlisting}

        \begin{enumerate}
            \item Створіть колекцію Tweets
                \begin{lstlisting}[language=python]

                \end{lstlisting}
        \end{enumerate}
        Серед інших атрибутів, Tweet має містити кількість лайків - атрибут likes
            \begin{lstlisting}[language=python]

            \end{lstlisting}

        \begin{enumerate}
            \item Сконфігуруйте шарди таким чином, щоб кожна з них була проасоційована з діапазоном кількості лайків (Ranged Sharding): Shard 1 - 0-100, Shard 2 - 101-200, Shard 3 - 200 - …
                \begin{lstlisting}[language=python]

                \end{lstlisting}
        \end{enumerate}
        І записи які будуть вставлятись в колекцію Tweets мають зберігатись на відповідній шарді в залежності від значення likes
            \begin{lstlisting}[language=python]

            \end{lstlisting}

        \begin{enumerate}
            \item Перевірте появи шард і зон командою sh.status()
                \begin{lstlisting}[language=python]

                \end{lstlisting}
            \item Продемонструйте роботу шардінгу (тобто що записи зберігаються на різних нодах):
                \begin{lstlisting}[language=python]

                \end{lstlisting}
            \begin{itemize}
                \item відключити одну з ноди
                    \begin{lstlisting}[language=python]

                    \end{lstlisting}
                \item спробувати додати записи зі значеннями shard key (Ranged та Zones), що потрапляють на відключену ноду
                    \begin{lstlisting}[language=python]

                    \end{lstlisting}
                \item спробувати додати записи зі значеннями shard key (Ranged та Zones), що потрапляють на працюючу ноду
                    \begin{lstlisting}[language=python]

                    \end{lstlisting}
                \item спробувати знайти всі записи з shard key для Zone, яка відповідає ноді яка працює/яка не працює
                    \begin{lstlisting}[language=python]

                    \end{lstlisting}
                \item спробувати знайти записи для shard key з певного проміжку, який входить до проміжку працюючої ноди для Ranged Sharding
                    \begin{lstlisting}[language=python]

                    \end{lstlisting}
                \item спробувати знайти записи для shard key з певного проміжку, який входить до проміжку ноди яка не працює для Ranged Sharding
                    \begin{lstlisting}[language=python]

                    \end{lstlisting}
            \end{itemize}
            \item Включити відключену ноду та перевірити працездатність запитів з попереднього пункту
                \begin{lstlisting}[language=python]

                \end{lstlisting}
        \end{enumerate}
        Всі завдання можуть бути виконані з командного рядка

        Протокол має містити команди та кінцеві налаштування з sh.status()
            \begin{lstlisting}[language=python]

            \end{lstlisting}
\end{document}
